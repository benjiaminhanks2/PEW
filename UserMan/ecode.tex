\documentstyle[11pt,A4]{article}

\newcommand{\SEX}[2]{
\begin{tabbing}
mmmmmmmmmmmmm\=\kill
{\bf Stack on #1}: $\rightarrow$\>
\+
#2\\
\end{tabbing}
}

\newcommand{\SQEX}[3]{
\begin{tabbing}
mmmmmmmmmmmmm\=\kill
{\em #3:}\\
{\bf Stack on #1}: $\rightarrow$\>
\+
#2\\
\end{tabbing}
}

\newcommand{\SE}[1]{\SEX{Entry}{#1}}

\newcommand{\SX}[1]{\SEX{Exit}{#1}}

\newcommand{\SQE}[2]{\SQEX{Entry}{#2}{#1}}

\newcommand{\SQX}[2]{\SQEX{Exit}{#2}{#1}}

\newcommand{\AC}[1]{
\begin{description}
\item[Action]\mbox{}\\
#1
\end{description}
}

\newcommand{\AL}[1]{
\begin{tabbing}
mmmmmmmm\=mmm\=mmm\=mmm\=mmm\=mmm\=mmm\=mmm\=\kill
{\bf Algorithm:}\>
\+
#1\\
\end{tabbing}
}

\begin{document}

{\Large The Estelle E-Code}

\section{The Stack}

The E-Code machine uses an array of 16-bit integers
called the code store, which holds the program code (of
fixed length). Each process is also allocated a stack
(also an array of 16-bit integers). Each stack begins at
address 1 and grows upwards.

Three index registers are used to access the stack:

\begin{itemize}
\item {\em p} -- (program) address of current instruction
\item {\em b} -- (base) start of stack (current activation)
\item {\em s} -- (stack) current stack pointer
\end{itemize}

Each process has its own {\em p}, {\em b} and {\em s} registers.

Initially the stack is empty. When execution begins,
space is allocated on the stack for the exported
variables, parameters, and variables defined in the
process. The exported variables occur first, then the
parameters, followed by a dummy activation record (see
below), and then the variables.

When the program activates a procedure, space is
allocated in an activation record; this is freed up on
exit from the procedure. Activation records consist of:

\begin{itemize}
\item Parameters
\item Context (static link, dynamic link, return address)
\item Variables
\item Temporary scratch area
\end{itemize}

The first three parts are of fixed length and are created
when the procedure is activated. The temporary part holds
operands and results during the execution of statements.
It is empty at the beginning and end of every statement.
The {\em b} register contains the address of the static link.
This is called the base address of the activation record.
Accesses to variables and parameters is done by using (positive and
negative) displacements from the {\em b} register.

The dynamic link of an activation record contains the
base address of the previous activation record. When a
procedure terminates, the dynamic link stored in the
current activation record is assigned to the {\em b} register.
These links represent the dynamic sequence in which
blocks are activated.

The static links define the set of variables that are
accessible within the current block (the current
context). These links represent the static block
structure of the program text.

To access a variable or parameter, we need to know the
activation record and displacement. To aid this, the
compiler assigns a level number to each block in a
program. During code generation, the compiler assigns a
level number and displacement to each variable. The
relative level of a block is the number of static links
that must be followed to reach a variable. Relative level
0 is the current activation record.

Accesses to variables, parameters, etc, are handled by
instructions which calculate the address of the variable
as an offset in the stack. A reference to an exported
variable of a child process results in a negative
address. By examining the address, a process can tell
whether the variable exists on its own stack, or the
stack of a child. In the latter case, the index of the
child process will have been pushed on to the stack as
well, so this can be popped, the correct child stack
identified, and the variable accessed (after first making
the address positive).

The use of negative addresses for exported variables is
an elegant solution to a complex problem, but has a side
effect - the address 0 can never be used. For this
reason, stack addressing starts at address 1. However,
this has an advantage, as it allows easy checking for
null pointer assignments - any attempt to access address
0 is an error.

We now turn to the instructions used in the program code.
We will first examine the Pascal P-code subset, and then
look at the extended E-code instructions. Throughout the
description, we assume that:

\begin{tabbing}
mmm\=cccccccccccccccc\=ccccccccccccccccc\=\+\kill
{\bf push(v)}\>{\em is the same as}\>{\bf stack[++s] = v}\\
{\bf pop(v)}\>{\em is the same as}\>{\bf v = stack[s--]}\\
{\bf chain(level,x)}\>{\em is the same as}\>{\bf x=b; while (level--)
x=stack[x]}\\
\end{tabbing}

Note that the P-code is not any particular standard; it
is adapted from the P-code used by Per Brinch-Hansen in
`Brinch-Hansen on Pascal Compilers'.

The format used in describing the operations is to give the stack
entry and exit states, where important, followed by an informal
description of the action of the instruction, and a pseudo-code
algorithm for the instruction. As we progress, less algorithms will
be given, as they will be assumed to be self-evident except in
complex cases (although in some complex cases, particularly the
E-Code instructions, algorithms are not given as they depend too much
on internal implementation details for processes, transitions, etc).
Sometimes several instructions are grouped together;
an algorithm is then given which is representative of all those in
the group. The stack states only show information which is
relevant to and altered by the instruction - it is assumed that
everything below this information remains unchanged.

\section{The Pascal P-Code Subset}

\subsection{Variable(Relative Level, Displacement)}
\SX{Absolute Address of Variable}
\AC{Used to access value parameters and local
variables. Compute the absolute address of a
variable and place it on the stack. This is done
by traversing the specified number of levels of
the static chain, obtaining the relative base
address, adding the displacement, and pushing the
result on to the stack.}
\AL{void variable(level,displ)\\
\{\\
\>chain(level,x);\\
\>push(x+displ);\\
\>p += 3;\\
\}}

\subsection{VarParam(Relative Level, Displacement)}
\SX{Address of variable bound to Parameter}
\AC{Used to access variable parameters. Variable
parameters are implemented by storing the address
of the bound variable as the value of the
parameter. Thus, to access the address of a
variable parameter, we need to find the address of
the parameter, and push its value (the address of
the bound variable) onto the stack. This is a
minor modification of the action of Variable.}
\AL{void VarParam(level, displ)\\
\{\\
\>chain(level,x);\\
\>push(stack[x+displ]);\\
\>p += 3;\\
\}}

\subsection{Index(Lower, Upper, Element size)}
\SE{ Array Index Value\\Base Address of Array}
\SX{Base Address + (EltSize * IndexValue)}
\AC{Used to access array elements. Removes an index
value from the top of the stack, checks that it is
in the specified range (if not, reports an error),
and increments the address value on the top of the
stack by the index value multiplied by the element
size; ie, the address of the indexed element. The
index must be checked for sign; if it is negative,
it represents a child's exported array. In this
case, the final address pushed must also be
negative.}
\AL{void Index(Lower, Upper, EltSize)\\
\{\\
\>exported = FALSE;\\
\>pop(ix);\\
\>if (ix$<$0) \{ exported = TRUE; ix = --ix; \}\\
\>if (ix$<$Lower {\tt ||} ix$>$Upper) Error;\\
\>else stack[s] += (ix--Lower)*EltSize;\\
\>if (exported) stack[s] = --stack[s];\\
\>p += 5;\\
\}}

\subsection{Field(Displ)}
\SE{Base Address of Record}
\SX{Base Address + Field Displacement}
\AC{Used to access fields of records. Adds the
specified field displacement to the address on top
of the stack. If the base address is negative, the
record is a child's exported record.}
\AL{void Field(Displ)\\
\{\\
\>if (stack[s]$<$0) stack[s] --= Displ;\\
\>else stack[s] += Displ;\\
\>p += 2;\\
\}}

\subsection{Constant(Value)}
\SX{Constant Value}
\AC{Pushes a constant value on to the stack.}
\AL{void Constant(Value)\\
\{\\
\>push(Value);\\
\>p += 2;\\
\}}

\subsection{Value(Length)}
\SE{Address of Variable\\{[Child number]}}
\SX{Value of Variable}
\AC{Used to access the value of a variable. Pops the
address on the top of the stack and pushes the
value held at that address (the number of words to
push is specified by the Length).}
\AL{void Value(Length)\\
\{\\
\>pop(address);\\
\>if (address$<$0)\\
\>\>\{\\
\>\>pop(child);\\
\>\>select child's stack;\\
\>\>address = --address;\\
\>\>\}\\
\>while (Length--\.--)\\
\>push(stack[address++]);\\
\>p += 2;\\
\}}

\subsection{Not(), Minus(), Abs(), Sqr(), Odd()}
\SE{Value}
\SX{Result of performing unary op on value}
\AC{Negates the (Boolean or Integer) value,
calculates the absolute value, calculates the
square root, or returns a Boolean 1 if the value is odd,
0 otherwise, respectively.}
\AL{void Not(void)\\
\{\\
\>stack[s] = !stack[s];\\
\>p++;\\
\}}

\subsection{Multiply(), Div(), Modulo(), Add(), Subtract(), And(),
Or()}
\SE{Value2\\Value1}
\SX{Value1 {\em op} Value2}
\AC{Multiplies (etc) the top two elements of the
stack, and replace them with the result.}
\AL{void Multiply(void)\\
\{\\
\>pop(val);\\
\>stack[s] *= val;\\
\>p++;\\
\}}

\subsection{Less(), Equal(), Greater(), NotGreater(), NotEqual(),
NotLess()}
\SE{Value2\\Value1}
\SX{Value1 {\em relop} Value2}
\AC{Replaces the top two values on the stack with a
Boolean TRUE if the first value is less (etc) than
the second one; otherwise FALSE.}
\AL{void Less(void)\\
\{\\
\>pop(val);\\
\>stack[s] = (stack[s] $<$ val);\\
\>p++;\\
\}}

\subsection{Assign(Length)}
\SE{Value to Assign\\Address to Assign to\\{[Child number]}}
\SX{}
\AC{Removes the value on the top of the stack, the
address below it, and assign the value to the
address.}
\AL{void Assign(Length)\\
\{\\
\>s --= length;\\
\>pop(rvalue);\\
\>lvalue = s+2;\\
\>if (rvalue$<$0)\\
\>\>\{\\
\>\>rvalue = --rvalue;\\
\>\>pop(child);\\
\>\>deststack = child stack;\\
\>\>\}\\
\>else deststack = stack;\\
\>while (Length--\.--)\\
\>deststack[rvalue++] = stack[lvalue++];\\
\>p += 2;\\
\}}

\subsection{Goto(Displ)}
\AC{Increment the p register by the displacement.}
\AL{void Goto(Displ)\\
\{\\
\>p += Displ;\\
\}}

\subsection{Do(Displ)}
\SE{Boolean Value}
\SX{}
\AC{Either proceed to the next instruction or jump to
another instruction, depending on the value on the
top of the stack (which is removed).}
\AL{void Do(Displ)\\
\{\\
\>if (pop()) p+=2;\\
\>else p+= Displ;\\
\}}

\subsection{For(Displ, Mode)}
\SE{End Value\\Start Value\\Address of Control Variable}
\SQX{loop unfinished}{End Value\\Address of Control Variable}
\SQX{loop finished}{}
\AC{Initialise a loop control variable with a start
value, compare it to an end value, and if the loop
is over, jump out using a displacement. Otherwise
push the address and end value back on to the
stack for use by the Next instruction. The Step
field is the STEP size, adjusted to be positive
(TO) or negative (DOWNTO).}
\AL{void For(Displ, Step)\\
\{\\
\>pop(end);\\
\>pop(start);\\
\>pop(address);\\
\>stack[address] = start;\\
\>if (Step*start $>$ Step*end)\\
\>p += Displ;\\
\>else \{\\
\>\>push(address);\\
\>\>push(end);\\
\>\>p+=3;\\
\>\>\}\\
\}}

\subsection{Next(Displ, Mode)}
\SE{End Value\\Address of Control Variable}
\SQX{loop unfinished}{End Value\\Address of Control Variable}
\SQX{loop finished}{}
\AC{Check whether or not a FOR loop is complete. If
not, leave stack as it was and bramnch back using
the Displ. If it is, leave stack clear and drop
through to next instruction. See also For.}
\AL{void Next(Displ, Step)\\
\{\\
\>pop(end);\\
\>pop(address);\\
\>stack[address]+=Step;\\
\>if (Step$>$0 \&\& stack[addr]$<$end) or\\
\>\>(Step$<$0 \&\& stack[addr]$>$end)\\
\>\>\{\\
\>\>push(address);\\
\>\>push(end);\\
\>\>p += Displ;\\
\>\>\}\\
\>else p+= 3;\\
\}}

\subsection{ProcCall(Level, Displ, ReturnLen)}
\SX{Return Address\\Dynamic Link\\Static Link}
\AC{Creates the context part of an activation record,
and jumps to the procedure code.}
\AL{void ProcCall(Level, Displ, ReturnLen)\\
\{\\
\>chain(Level,x);\\
\>s += ReturnLen;\\
\>push(x); /* static link */\\
\>push(b); /* dynamic link */\\
\>push(p+3); /* return address */\\
\>b = s--2;\\
\>p += Displ;\\
\}}

\subsection{Procedure(VarLength, TempLength, Displ)}
\SX{Last Word of Local Variable Space\\Local Variable Space}
\AC{Allocates space on the stack for a procedures
variables, ensures that there is sufficient space
remaining for the temporaries, and branches to the
statement part of the procedure.}
\AL{void Procedure(VarLen, TmpLen, Displ)\\
\{\\
\>if ((s+VarLen+TmpLen)$>$stacksize) expand stack;\\
\>s += VarLen;\\
\>p+= Displ;\\
\}}


\subsection{EndProc(ParamLength, ReturnLen)}
\SE{ Last Word of Local Variable Space\\Local Variable Space\\
Activation Record\\{[Return Value]}\\Parameters}
\SX{[Return Value]}
\AC{Uses the dynamic link to remove the activation
record of the procedure and jump to the return
address, and pushes the return value, if any.}
\AL{void EndProc(ParamLen)\\
\{\\
\>s = b+2; /* pop variables */\\
\>ReturnValTop = b--1;\\
\>pop(p);\\
\>pop(b);\\
\>pop();\\
\>s --= ParamLen+ReturnLen;\\
\>while (ReturnLen--\.--)\\
\>\>push(stack[ReturnVarTop--ReturnLen]);\\
\}}

\subsection{ReadStr(Length), ReadCh(), ReadInt()}
\SE{Address to Read to\\{[Child number]}}
\SX{}
\AC{Input a string, character or integer and assign
it to the address on the top of the stack. System
dependent!}
\AL{void ReadStr(Length)\\
\{\\
\>Read string from keyboard;\\
\>pop(address);\\
\>if (address$<$0)\\
\>\>\{\\
\>\>address = --address;\\
\>\>pop(child);\\
\>\>select child's stack;\\
\>\>\}\\
\>while (Length--\.--)\\
\>\>stack[address++] = next character of string;\\
\>p+=2;\\
\}\\
\\
void ReadInt() /* ReadCh similar */\\
\{\\
\>Read integer from keyboard;\\
\>pop(address);\\
\>if (address$<$0)\\
\>\>\{\\
\>\>address = --address;\\
\>\>pop(child);\\
\>\>select child's stack;\\
\>\>\}\\
\>stack[address] = integer;\\
\>p++;\\
\}}


\subsection{WriteStr(Length), WriteCh(), WriteInt()}
\SE{Value to Write}
\SX{}
\AC{Outputs the value on the top of the stack. System
Dependent!}
\AL{void WriteStr(Length)\\
\{\\
\>s --= Length;\\
\>Output Length characters from stack[s+1] on.\\
\>p+=2;\\
\}\\
\\
void WriteInt(void) /* WriteCh is similar */\\
\{\\
\>pop(value);\\
\>Output value as an integer;\\
\>p++;\\
\}}

\subsection{EndWrite()}
\AC{Indicates the end of a WRITE or WRITELN
statement. The action is system-dependent; in the
PEW this results in the I/O window popping up and
being displayed for one second.}

\subsection{DefArg(Label,Value)}
\AC{Inform the linker of the actual value of a
symbolic label, ie L[Label] = Value. These
instructions are removed from the code output by
the linker.}

\subsection{DefAddr(Label)}
\AC{Inform the linker of the actual address of a
symbolic label, ie L[label] = current address.
These instructions are removed from the code
output by the linker.}

\subsection{AddSetElement()}
\SE{Element Number (0\dots 63)\\Set Value (4 words)}
\SX{New Set Value (4 words)}
\AC{The specified element is added to the set. Sets
are represented using 64-bits (ie, four 16-bit
words), and the element number is a value in the
range 0\dots 63. This specified bit is set in the set
value.}
\AL{void AddSetElement(void)\\
\{\\
\>pop(elt);\\
\>word = elt/16;\\
\>bit = elt\%16;\\
\>stack[s--word] {\tt |}= 1$<<$bit;\\
\>p++;\\
\}}


\subsection{AddSetRange()}
\SE{Range End (0\dots 63)\\Range Start (0\dots 63)\\Set Value (4 words)}
\SX{New Set Value (4 words)}
\AC{All bits in the specified range are set. See
AddSetElement (41) for information on the
representation of sets.}
\AL{void AddSetRange(void)\\
\{\\
\>pop(end);\\
\>pop(start);\\
\>while (start$<$=end)\\
\>\>\{\\
\>\>word = elt/16;\\
\>\>bit = elt\%16;\\
\>\>stack[s--word] {\tt |}= 1$<<$bit;\\
\>\>start++;\\
\>\>\}\\
\>p++;\\
\}}

\subsection{In()}
\SE{Stack Value (4 words)\\Element Number (0\dots 63)}
\SX{Boolean result}
\AC{If the specified bit is set in the set value,
push 1 on to the stack, else push 0 on to the
stack.}
\AL{void In(void)\\
\{\\
\>s --= 4;\\
\>pop(elt);\\
\>word = elt/16;\\
\>bit = elt\%16;\\
\>if (stack[s+5--word] \&\& (1$<<$bit)) push(1);\\
\>else push(0);\\
\>p++;\\
\}}

\subsection{SetAssign()}
\SE{Stack Value (4 words)\\Destination address\\{[Child number]}}
\SX{}
\AC{Assigns the four words holding the set value to
the four words starting at the specified
destination address. If the destination address is
negative, the variable is a child's exported
variable, so the child number is used to determine
which child stack to use.}
\AL{void SetAssign(void)\\
\{\\
\>dest = stack[s--4];\\
\>source = s--4;\\
\>if (dest$>=$0) deststack = current stack;\\
\>else \{\\
\>\>pop(child);\\
\>\>deststack = child's stack;\\
\>\>\}\\
\>while (source!=s)\\
\>\>deststack[dest++] = stack[source++];\\
\>s --= 5;\\
\>p++;\\
\}}

\subsection{SetEqual()}
\SE{Set1 (4 words)\\Set2 (4 words)}
\SX{Boolean value}
\AC{Pushes 1 on the stack if the two sets are
identical.}
\subsection{SetInclusion, SetInclusionToo}
\SE{Set1 (4 words)\\Set2 (4 words)}
\SX{Boolean value}
\AC{Pushes 1 on the stack, if Set2 is a subset of Set1
(SetInclusion), or Set1 is a subset of Set2 (SetInclusionToo),
else pushes 0.}

\subsection{Intersection(), Union(), Difference()}
\SE{Set1 (4 words)\\Set2(4 words)}
\SX{Result Set (4 words)}
\AC{Computes the union, intersection or difference (Set2--Set1) of two
sets, and leaves the result on the stack.}

\subsection{StringEqual(Len), StringGreater(Len), StringLess(Len),
StringNotEqual(Len), StringNotGreater(Len), StringNotLess(Len)}
\SE{String1 (of length `Len')\\String2 (of length `Len')}
\SX{Boolean value}
\AC{Pushes a 1 on to the stack if String2 satisfies the specified
relationship with String1, else pushes 0. For example, if String1 is
gretaer than String2, then StringLess would push 1. `Len' specifies
the length of the strings.}

\subsection{IndexedJump(Offset)}
\AC{This is used for CASE statements. It should be followed by a
number of GOTO instructions, which correspond to the different cases.
The offset is popped, multiplied by 2 (the size of a GOTO
instruction), and added to the p register to get the new instruction,
which should be a GOTO.}

\subsection{Case()}
\AC{Generates a run-time error. This instruction is used to signal
the end of a case statement, and corresponds to the situation where
the case expression has no corresponding label.}

\subsection{Copy()}
\SE{Value}
\SX{Value\\Value}
\AC{Copy duplicates the value on the top of the stack.}

\subsection{Pop()}
\SE{Value}
\SX{}
\AC{Pop the value from the top of the stack.}

\subsection{New(Numwords)}
\SX{Address of Allocated Block}
\AC{Attempts to find an unallocated block of memory on the heap (free
stack space) of the specified number of words. In the {\em PEW}, if such
a block is
found, it is linked in to a chain of allocated blocks, and the
address is pushed on to the stack. If the allocation is not possible,
a run-time error occurs. The use of a chain of allocated blocks is
inefficient - it would be better to use a chain of free blocks;
however, such details are implementation dependent. If the New
instruction is the first such instruction to be executed by the
containing process, its stack is expanded before allocation. Once a
New has occurred, a process' stack is fixed in size and cannot be
re-expanded.}

\subsection{Dispose()}
\SE{Address of Allocated Block}
\SX{}
\AC{Locates the allocated block on the chain of allocated blocks (see
`New'), and removes it from the chain. If no corresponsing block is
found, a run-time error occurs.}

\subsection{EnterBlock(Stack Space)}
\AC{Specifies the minimum amount of stack space that is required to
execute the block about to be entered. A check is performed to see if
the stack has sufficient free space, and if not, an attempt is made
to expand the stack. If this fails, a run-time error is reported.}

\subsection{Newline(Linenum)}
\AC{Used to synchronise the E-Code with the original Estelle source
code.}

\subsection{Random()}
\SE{Range of Random Numbers}
\SX{Random Number}
\AC{Pops the Range value off the stack, generates a
random number between {\em 0} and {\em Range--1} inclusive, and push
this value on the stack.}

\subsection{Range(Minimum, Maximum)}
\SE{Value}
\SX{Value}
\AC{Checks that the value on the top of the stack is
in the specified range; if not, generates a run-time
error.}

\subsection{ReturnVar(Displ)}
\SE{}
\SX{Address of Variable}
\AC{Adds the displacement to the activation record
base address to get the return variable address, and
pushes this on to the stack.}

\subsection{Scope(Level,Offset)}
\AC{Used to keep track of symbol table scope. The parameters are the
scope level, and offset into the symbol table storage area of the
first symbol table entry at this level. This is highly implementation
dependent.}

\subsection{With()}
\SE{}
\SX{Activation Record for WITH statement}
\AC{Used to identify the new scope level of a WITH statement. It
simply pushes a new activation record on to the stack.}

\subsection{WithField(Field Offset, With Level)}
\SE{}
\SX{Address of Field}
\AC{References to fields within WITH statements cause this
instruction to be generated. The With Level is the number of
activation records to be traversed to get to the record address
associated with the field, while the Field Offset is the offset of
the field within that record.}
\AL{void WithField(FieldOffset, WithLevel)\\
\{\\
\>chain(WithLevel,addr);\\
\>recaddr = stack[addr--1];\\
\>if (recaddr$<$0) fldaddr = recaddr -- FieldOffset;\\
\>else fldaddr = recaddr + FieldOffset;\\
\>push(fldaddr);\\
\}}

\subsection{EndWith(Numlevels)}
\AC{Pops the specified number of activation records off the stack
(each record in the list of a WITH statement causes an activation
record to be pushed on to the stack, {\bf preceded} by the address of
the record being specified), and then pops the address of the first
record.}

\section{E-Code extensions to P-Code}

\subsection{NumChildren()}
\SX{Number of Module Variables}
\AC{Pushes the number of module variables that the currently executing
process has on to the stack.}

\subsection{GlobalTime()}
\SX{Current Time}
\AC{Pushes the current simulation time on to the stack. WARNING - the
time is pushed as a 16-bit word, whereas times are actually
represented in the {\em PEW} as 32-bits. Truncation is thus a very
real possibility.}

\subsection{FireCount()}
\SX{Number of Executions}
\AC{Pushes the number of times the currently executing transition has
been executed on to the stack.}

\subsection{QLength()}
\SE{IP Index\\IP Identifier}
\SX{IP Queue Length}
\AC{Pushes the length of the reception queue associated with the
specified IP on to the stack. The IP Identifier is not used, but is
generated by the compiler for IP references anyway, so it is simply
discarded.}

\subsection{SetOutput(Directive Value)}
\AC{Implements the {\em PEW} compiler directive for controlling the
destination of output due to WRITE and WRITELN procedure calls.}

\subsection{Domain(Numvars)}
\SE{}
\SX{Activation Record\\Space for Domain Variables}
\AC{Used for EXIST, FORONE and FORALL constructs in Estelle. It
allocates space on the stack for the specified number of domain variables,
and then pushes a new activation record.}

\subsection{EndDomain(NumVars)}
\SE{Result\\Activation Record\\Domain Variables}
\SX{Result}
\AC{Identifies the end of a EXIST, FORONE or FORALL Estelle
construct. The result is popped off the stack, the activation record
popped, the domain variables released, and the result pushed back on
to the stack.}

\subsection{ExistMod(Header Ident, Offset to Failure Code)}
\SE{\dots\\Domain Activation Record\\Module Domain Variable}
\SX{\dots\\Domain Activation Record\\Module Domain Variable}
\AC{Checks whether the child process identified by the module domain
variable has the specified header identifier. If it does, control
falls through to the next statement. If not, control passes to the
failure code. This is used for domain constructs (EXIST, FORONE,
FORALL), where the domain type is a module domain. For example, to
execute an EXIST factor, the module domain variable is iterated from
0 through to {\em numchildren--1}, and ExistMod is executed for each
value, until all values are exausted, or a successful value is found.}

\subsection{ExpVar(Level,Displacement)}
\SE{Child Number}
\SX{Exported Variable Address\\Child Number}
\AC{Specifies a reference to an exported variable - the alter-ego of
the Variable instruction. The activation chain is traversed on the
{\em child's} stack to find the address.}
\AL{void ExpVar(Level,Displ)\\
\{\\
\>pop(childnum);\\
\>select child's stack;\\
\>chain(Level,x);\\
\>select parent's stack;\\
\>push(childnum);\\
\>push(--x--Displ);
\}}

\subsection{DefIPs(Start,Size)}
\AC{Used to specify which IP's share a common reception queue. The
linker rearranges the E-Code so that all DefIPs instructiosn for a
process occur contiguously. This block of instructions is then
processed when a module is created. The parameters specify the start
index of the set of IPs, and the number of IPs in the set. For
example, if a process has 7 IPs, and of these the 3rd, 4th and 6th
share a common reception queue, two instructions would be generated,
namely DEFIPS(4,2) and DEFIPS(7,1) (the numbering of IPs begins with
0).}

\subsection{Attach()}
\SE{Child IP Index\\Child IP Identifier\\Child Index\\
Parent IP Index\\Parent IP Identifier}
\SX{}
\AC{Attaches the child external IP to the parent IP. Any interactions
queued at the parent IP are prepended onto the queue associated with
the {\em downattach} of the child IP. For details on the semantics of
Attach, refer to the Estelle literature. This instruction corresponds
exactly to the high level ATTACH instruction in its effects.}

\subsection{Connect(IsInternal1, IsInternal2)}
\SE{IP Index1\\IP Ident1\\{[ModVar Index1 if IP1 is external]}\\IP Index2\\
IP Ident2\\{[Modvar Index2 if IP2 is external]}}
\SX{}
\AC{Connect the two specified interaction points. The
indices of the interaction points are on the stack. If
either or both of the interaction points are not internal
(ie, they are child external), then the index of the
appropriate child module variable must also be on the
stack.}

\subsection{Output(Length of Arguments, Reliability)}
\SE{[Interaction Arguments]\\Interaction Length\\Interaction
Identifier\\IP Index\\IP Identifier}
\SX{}
\AC{Pops and saves the Interaction Arguments, creates
a new interaction with these saved arguments and the
other parameters on the stack, timestamps the
interaction, and enqueues it on the appropriate queue
specified by the Interaction Point Index. The reliability is used to
randomly determine whether it should actually be output, or whether
it should be lost.}

\subsection{Detach(Mode)}
\SQE{if Mode is 0}{IP Index\\IP Identifier}
\SQE{if Mode is 1}{IP Index\\Child Index\\IP Identifier}
\SQE{if Mode is 2}{Child Index}
\SX{}
\AC{The Mode specifies whether all child external IPs of a specified
child are to be detached (mode 2), a specific child external IP is to
be detached (mode 1), or a specific IP belonging to the current
process is to be detached (mode 0). The IP Identifier is not used in
any case, but is discarded if present. The appropriate module
(current or child) is selected, and attachments of IPs for that
module are broken - either a single specific attachment, or all
attachments (mode 2). Interactions that arrived through the higher IP
are removed from the {\em downattach} of the lower IP and returned to
the queue of the higher IP.}

\subsection{Disconnect(Mode)}
\SQE{if Mode is 0}{IP Index\\IP Identifier}
\SQE{if Mode is 1}{IP Index\\Child Index\\IP Identifier}
\SQE{if Mode is 2}{Child Index}
\SX{}
\AC{The Mode is as for Detach. This instruction is very similar to
Detach, except that no interactions are moved.}

\subsection{Module(Body Identifier,Header Identifier, Class,
Variable Space, Number of Module Variables,
Number of Initialisation Transitions, Number of Body Transitions,
Number of Internal IPs, Number of External IPs, Offset to first
Initialisation Transition, Offset to first Body Transition, Offset to
DefIP instructions, Offset to Scope instructions)}
\AC{This is the instruction that initialises modules. Together with
TRANS, it is the modt complex instruction in the instruction set.

Various data structures for the new process are created, including a
module variable table, transition table, IP table, and stack. Reception
queues for the IPs are alloacted, with the DefIPs instructions being
processed to determine which IP table entries share the same
reception queue. If the Offset to DefIPs instructions has the value
--1, there are no common queues.

A scope table is allocated for the module, and the Scope instructions
are processed to load up this table.

The initialisation parameters are popped off the parent stack (see
Init) and pushed on to the new stack. Space is reserved for the
exported variables, an activation record is created, and then space
is reserved for the (non-exported) variables.

If the module has initialisation transitions, one of these is
selected for execution and executed. Finally, the parent process is
reactivated, and execution continues from the parents Init
instruction.}

\subsection{Init(Displacement to Module instruction, ParamLen)}
\SE{[Initialisation Parameters]\\Module Variable Index}
\SX{}
\AC{Creates a new module instance, using the module
body definition at the specified address (see Module), and assigns the
new module instance to the specified module variable.}

\subsection{Release()}
\SE{Child Number}
\SX{}
\AC{The specified child process and all its children are killed.
Before releasing the process, all its external interaction points are
disconnected and/or detached. Any module variables of the current
process that referred to the released process are made undefined.

Additional processing that occurs in the PEW is for all breakpoints
that refer to the released process or its children only are cleared
(not yet implemented), and the execution summary statistics for the
process and its children are written to the log file.}

\subsection{Terminate()}
\SE{Child Number}
\SX{}
\AC{Terminate is a simple form of Release. The only difference is
that external IPs of the terminated process that are attached at the
time of termination are detached using a form of detach known as {\em
simpledetach}. Unlike detach, a simpledetach does not result in any
interactions being moved.}

\subsection{ModuleAssign()}
\SE{Source Module Variable Number\\Destination Module Variable
Identifier\\Destination Module Variable Number}
\SX{}
\AC{Assigns the destination module variable to have the same value as
the source. The identifier is used in the {\em PEW} for symbolic support
by the process browser.}

\subsection{Delay(Type)}
\SE{Minimum Delay\\Maximum Delay}
\SX{Boolean Value}
\AC{Compares the length of time that the transition
has been enabled (call this {\em etime} with the Maximum
Delay. If the etime exceeds the maximum delay, TRUE is pushed on to the
stack. If the etime is less than the Minimum Delay, FALSE
is pushed on to the stack. If the etime falls between
the minimum and maximum delay times, a random delay in the interval
is chosen (according to the specified distribution), and if this is
less than the etime TRUE is pushed, else FALSE is pushed. The {\em
PEW} also keeps track of the shortest delay over all DELAY clauses in
a specification, and if no transitions
are enabled but the system has not deadlocked, the global time is
updated by this shortest delay.

The Type parameter specifies the distribution type for random delays -
if the type is 0 then a uniform distribution is used; if the type is 
negative a Poisson distribution (with --Type as the mean value) is used;
otherwise an exponential distribution with Type as the mean value is
used.}

\subsection{When(Interaction Identifier, Interaction Length}
\SE{IP Offset}
\SX{Value}
\AC{The identifier of the interaction at the head of the specified IP
is compared with the Interaction Identifier argument. If they are
identical, and the destination IP of the head interaction is indeed
the specified IP (this check is necessary because of common queues)
then (IP Offset + 1) is pushed on to the stack, else 0 is
pushed on to the stack. The +1 is necessary to ensure a non-zero
value, as IP Offsets begin at zero. The `len' parameter is not used,
and will be removed in the future.}

\subsection{Otherwise(First,Last)}
\SE{}
\SX{Boolean Value}
\AC{This instruction implements the PROVIDED OTHERWISE construct. The
parameters First and Last specify the range of transitions whose
PROVIDED clauses must all have failed for the PROVIDED OTHERWISE
clause to succeed. As the transitions in the range must all precede
the transition containing the PROVIDED OTHERWISE clause, the state of
their provided clauses is known by the time this instruction
executes.}

\subsection{Trans(Name, FromStates, From Identifier, ToState, To
Identifier, Priority, Offset to Provided Clause, Offset to When
Clause, Offset to Delay Clause, Offset to next Trans instruction,
Length of When Interaction, Size of Variable Space, Size of Temporary
Space, Offset to Transition Block)}
\AC{This instruction, like Module, is not really `executed', as such,
but is used to impart information to the interpreter about a
transition. The first time that the scheduler examines the
containing process, the identifiers, name, and priority are all
entered into the transition table entry for the transition. The
scheduler will also use the Trans instruction each time to locate the
clauses of the transition, and if the transition is selected for
execution, to locate its body. The Variable and Temporary Space
arguments are used to check for potential stack overflow before
executing a transition.}

\subsection{EndTrans()}
\AC{Indicates the end of a transition block. No
action is taken.}

\section{P-Code Optimisations}

Some of the P-Code instructions are optimised forms of common P-Code
instruction pairs. These are:
\begin{tabbing}
cccc\=ccccccccccccccccccccc\=\+\kill
{\em Instruction:}\>{\em is equivalent to:}\\
LocalVar(Displ)\>Variable(0,Displ)\\
LocalValue(Displ)\>Variable(0, Displ); Value(1)\\
GlobalVar(Displ)\>Variable(1,Displ)\\
GlobalValue(Displ)\>Variable(1,Displ); Value(1)\\
SimpleValue\>Value(1)\\
SimpleAssign\>Assign(1)\\
GlobalCall(Displ)\>ProcCall(1,Displ)\\
\end{tabbing}

Furthermore, a sequence:
\begin{verse}
{\bf Variable(Level, VDispl)}\\
{\bf Field(FDispl)}
\end{verse}
can be replaced by the single instruction:
\begin{verse}
{\bf Variable(Level, VDispl+FDispl)}
\end{verse}

\section{The Scheduler}

This is the heart of the interpreter. It is an `infinite' loop, which
successively executes the following steps:
\begin{enumerate}
\item{Calculate, recursively, for all processes in the process tree,
the state of each clause of each transition of each process. Keep
track of the smallest positive delay of all transitions that are only
disabled due to their delay clauses failing.}
\item{If no enabled transition were found, update the time by the
smallest delay found, and start again.}
\item{If enabled transitions were found, once again recurse through
the process tree. If a process has no enabled transitions, recurse
through its children, else bootom out the recursion process at this
point (as parent transitions always take priority). If the process
offers more than one transition to be fired, use the priorities to
select. If there are transitions that are enabled and have the same
priorities, select randomly between them.}
\item{When recursing through children, if the parent is an activity
or systemactivity, bottom out of the recursion the moment a single
child has a transition executed. For processes and systemprocesses,
continue the recursion through all of the children's peers.}
\item{If no transitions were found, and no delay was found in the
first step, the system has deadlocked.}
\end{enumerate}
\end{document}
